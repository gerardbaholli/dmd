\section{Directory Structure}
The files in the folder are divided by theme, if you want to access information about a particular character just go to the relating sub-folder of that character and you will find all the material.
The structure of the directory must be preserved to allow non-dispersion of the files. It is also important not to create multiple copies of the same file located in different paths, this to make the directory easily accessible to other users.

\vspace*{0.5cm}
\dirtree{%
	.1 repository.
	.2 characters.
	.3 bad\_eleven.
	.4 2D\_models.
	.4 sounds.
	.3 ....
	.2 enemies.
	.3 common.
	.4 2D\_models.
	.4 sounds.
	.3 bosses.
	.4 ....
	.2 levels.
	.3 01\_level.
	.4 environments.
	.4 sounds.
	.4 story\_dialogues.
	.5 dialogues.
	.5 extras.
	.3 ....
	.2 props.
	.4 2D\_models.
	.2 documentation.
	.3 milestones.
	.4 01\_milestone.
	.4 02\_milestone.
	.4 03\_milestone.
	.4 04\_milestone.	
	.3 DMD\_source.
	.3 LDD\_source.
	.3 data\_management\_document.pdf.
	.3 level\_design\_document.pdf.
}

\vspace*{0.5cm}
In this tree the main folders are displayed but in them there are other sub-folders that contain the files and keep them in order, distinguishing them by level or by type.